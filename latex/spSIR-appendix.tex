\documentclass[english,11pt]{article}
%\usepackage[nomarkers]{endfloat}
\usepackage{babel}
\usepackage[T1]{fontenc}
\usepackage[latin9]{inputenc}
\usepackage{geometry}
\geometry{verbose,lmargin=1.5in,rmargin=1.5in}
\usepackage{color}
\usepackage{array,multirow,tablefootnote,float,booktabs,makecell,longtable}
\usepackage{nicefrac}
\usepackage{caption}
\captionsetup{labelfont=bf,textfont=bf,position=auto}
\usepackage[flushleft]{threeparttable}
\usepackage{amstext}
\usepackage{amsthm}
\usepackage{graphicx}
\usepackage{enumitem}
\usepackage{titlesec}
\usepackage{subcaption}
\titleformat{\subsubsection}[runin]{\bfseries}{}{}{}[]
\usepackage[authoryear]{natbib}
\usepackage{overpic}

%%%% THIS is to make xr package work in overleaf
\usepackage{xr-hyper}
\makeatletter
\newcommand*{\addFileDependency}[1]{% argument=file name and extension
  \typeout{(#1)}
  \@addtofilelist{#1}
  \IfFileExists{#1}{}{\typeout{No file #1.}}
}
\makeatother
\newcommand*{\myexternaldocument}[2][]{%
    \externaldocument[#1]{#2}%
    \addFileDependency{#2.tex}%
    \addFileDependency{#2.aux}%
}
\myexternaldocument[main:]{spSIR-short-revision}

\usepackage[unicode=true,pdfusetitle,
 bookmarks=true,bookmarksnumbered=true,bookmarksopen=true,bookmarksopenlevel=2,
 breaklinks=true,pdfborder={0 0 0},pdfborderstyle={},backref=page,colorlinks=true]
 {hyperref}
\hypersetup{citecolor=darkblue,linkcolor=darkblue,urlcolor=darkblue}
\definecolor{darkblue}{rgb}{0,0.1,0.5}

%%%% THIS IS FOR BACK REFERENCES IN BIBLIO
\renewcommand*{\backref}[1]{}
\renewcommand*{\backrefalt}[4]{{\ifcase #1%
          \or(Cited on page~#2)%
          \else(Cited on pages #2)%
    \fi%
    }}
    

%%%%%%%%%%%%%%%%%%%%%%%%%%%%%%%%%%%%%%%%%%%%%%%%%%%%%%%%%
\begin{document}
\setcounter{footnote}{3}
\author{
    Alberto Bisin 
    \and 
    Andrea Moro}

\title{Learning Epidemiology by Doing: \\ The Empirical Implications of a Spatial-SIR Model with Behavioral Responses: External Appendix\thanks{Please check our websites
for an updated version of this paper. 
Bisin: New York University, \href{https://wp.nyu.edu/albertobisin/}{\texttt{wp.nyu.edu/albertobisin/}}, 
\texttt{alberto.bisin@nyu.edu}. Moro: Vanderbilt University, \href{https://andreamoro.net}{\texttt{andreamoro.net}}, \texttt{andrea@andreamoro.net}.}
}
\date{\today \\ {\small First version: June 10, 2020}}
\maketitle


\appendix
\renewcommand\thefigure{\thesection.\arabic{figure}}    

%\footnote{The local interaction model cannot be solved for analytically; see \ref{sec:app_theorySS}.}

\section{Appendix: Theoretical Structure of SIR and Spatial-SIR } \label{sec:app_theory}

In this appendix we construct the theoretical structure of SIR and Spatial-SIR as Markov chains processes. We present the structure of these models in discrete time first, for consistency with the simulation analysis in the text.

\subsection{SIR}  The society is populated by $N$ individuals. Agents are ex-ante identical in terms of demographic characteristics. Let $\mathcal{S}$ denote the individual state-space. In the SIR model, the state-space is $\mathcal{S}=\{S, I, R\}$, indicating Susceptibles, Infected, and Recovered. 
Let  $h^i_t \in \mathcal{S}$ denote the state of agent $i$ at time $t$. Let $h_t=\frac{1}{N}[S_t,I_t,R_t] \in \Delta^{\mathcal{S}}$ denote the distribution of the population across the state-space.\footnote{Abusing notation, we let we let the capital letters indicating a state also denote the fraction of the population in that state; and we let  $\mathcal{S}$ denote both the set and its numerability.}
The SIR model is represented by a Markov Chain: 
$$ prob(h^i_{t+1}=h' \mid h^i_t=h)= T_{h \: h'}(h_t)$$ where $T_{h \: h'}(h_t)$ is the generic element of a  $\mathcal{S} \times \mathcal{S}$ double-stochastic (transition) matrix $T(h_t)$. The dependence of the transition matrix on $h_t$, the distribution of the population across the state-space (the aggregate state of the economy), is a mean-field property justified in this class of models by random matching in the population.  

More specifically, the matrix $T_{h \: h'}(h_t)$ is determined by the following transitions: 

S $\longrightarrow$ I. A Susceptible agent becomes infected
 upon contact with an Infected, with probability $\pi I$. 

A $\longrightarrow$ R. An  agent Infected at $t$, at any future period, can  Recover with probability $\rho$.
%An Asymptomatic agent infected at $t$, if he/she does not transition
%away after $\overline{\tau}$ days, Recovers with certainty.

$R$ Recovered is absorbing state of the dynamic process (agents entering this state never leave). This assumes Recoved agents are immune to infection. \newline 

 The resulting dynamical system for the distribution of the population across the state-space, $h_t$, is the following, 

$$ h_{t+1}=T(h_t)h_t.$$
The dynamical system can be solved for in closed form, see e.g.,  \cite{Moll2020}, \cite{neumeyer2020clase}.
%Given any initial conditions specifying all agents Susceptible except an  initial fraction of Infected $I_0>0$, the dynamical system converges to a unique stationary state, with $R^* >0$. This stationary state is characterized in terms of $R_{0}$, a parameter of the model  denoting the number of agents a single infected agent
%infects, on average, at time $t=0$. Then at the stationary state, $R^*$ is obtained as the solution of the following fixed point
% equation:\footnote{See e.g., \cite{Moll2020} for an analytical solution of the SIR model. }
% \begin{equation}
% R^*+D^*=-\frac{1}{R_{0}}\ln(1-R^*-D^*).\label{ss2}
% \end{equation}
%\newline 

%\subsubsection{Theoretical Representation}

\subsection{Spatial-SIR \label{sec:app_theorySS}} We now add a spatial dimension to the SIR model. We also expand the state space to better caputure several relevant aspects of the  SARS-CoV-2 infection. Specifically, we split the $I$ state into Asymptomatics and sYmptomatics, $A$ and $Y$. We also add explicitly the state $D$, for Dead. Hence,  $\mathcal{S}=\{S, A, Y, R, D\}$. We maintain the notation  $h^i_t \in \mathcal{S}$ to denote the state of agent $i$ at time $t$; and $h_t=\frac{1}{N}[S_t,A_t,Y_t,R_t,D_t] \in \Delta^{\mathcal{S}}$ to denote the distribution  the $N$ agents in the  population across the state-space. 

Agents are  located in space, e.g., a lattice, which we call "the City." Agents are ex-ante identical in terms
of demographic characteristics and symmetric in terms of location in space. %a city, represented
%by a unit square. Agents can be in five different states: 
%\begin{quotation}
%Susceptible, infected Asymptomatic, infected sYmptomatic, Dead, or
%Recovered (S,A,Y,D,R). 
%\end{quotation}
A (Markov) transition process between states governs the dynamics
of the system from the initial condition, at day $t=0$. The spatial dimension maps the stochastic process into a local interaction model, a model in which agents' contacts are not the results of random matching but rather of local matching, with agents close in  space (geographical distance as a  metaphor for social distance).   Let $H_t $ denote the configuration of agent at time $t$, a vector $\left[h^1_t, h^2_t, \ldots, h^I_t\right]$; the set of all configuration is denoted $ \mathcal{H}$. 
The local interaction model is characterized by 
$$ prob(h^i_{t+1}=h' \mid h^i_t=h)= T_{h \: h'} ( H_t). $$ 

More specifically, the matrix $T_{h \: h'} ( H_t)$ is determined by the following transitions: 
%Agents in sYmptomatic and Dead state do not move. 

S $\longrightarrow$A. Susceptible agents become infected
 upon contact with an Asymptomatic, with probability $\pi$.\footnote{Susceptible agents are not infected upon contact with a sYmptomatic
agent; this is to capture the fact that sYmptomatic agents are either
isolated at home or in the hospital. They movements in the City are
vacuous.}  A contact is defined to occur when agents are at a geographical distance in space $\leq p$.

A $\longrightarrow$ Y, R. An Asymptomatic agent infected
at $t$, at any future period, can become %stays in state A at for $\tau^{Y}$ periods.
%starting at $t+\tau^{Y}+1$ he/she can become
sYmptomatic with probability $\nu$, or can Recover with probability $\rho$.
%An Asymptomatic agent infected at $t$, if he/she does not transition
%away after $\overline{\tau}$ days, Recovers with certainty.

Y $\longrightarrow$ R, D. An agent who has become sYmptomatic
at $t$, at any future period, can Recover with probability $\rho$, or can  Die with probability $\delta$. % every day from $t+3$ on.
%An agent who has become sYmptomatic at $t$, if he/she does not transition
%away after $\overline{\tau}$ days, Recovers with certainty.

D, R. Dead and Recovered are absorbing states of the dynamic process. As we noted, this assumes Recoved agents are immune to infection. \newline 

Abusing notation, a transition matrix $T( H_t)$ in the space of possible configurations $\mathcal{H}$ can be constructed from $T_{h \: h'} ( H_t)$.\footnote{This is an ugly looking operation, but formally straighforward, as purely arithemetical.} The resulting dynamical system for configurations $H_t$ is 
$$H_{t+1} =T( H_t)H_t.$$

But Spatial-SIR accounts for  agents possibly coming  into contact after moving randomly in space.\footnote{This is different from most mathematical literature on local interactions; see e.g., \cite{kindermann1980american} and \cite{liggett2012interacting}.} 
%\footnote{Let $l$ index a location in a finite space $\mathcal{L}$ and let $I_t(l)$ be the set of agents in location $l$ at time $t$.}
Let the operator $P_t$, mapping $H_t \in \mathcal{H}$ into $P_t \circ H_t \in \mathcal{H}$, represent a configuration after a random permutation of the position of the agents, indexed by $i$. Before transitioning from the state at $t$ to the state at $t+1$ the agents' locations are permutated randomly. The local interaction model is characterized by 
$$ prob(h^i_{t+1}=h' \mid h^i_t=h)= T_{h \: h'} ( P_t \circ H_t). $$  
The resulting dynamical system for configurations $H_t$ is:\footnote{This representation is complicated in that the state space $\mathcal{H}$ is very large, and the permutation does not help. A simpler representation of $ prob(h^i_{t+1}=h' \mid h^i_t=h)$ can be obtained as follows. Let $I_t$ map locations $l \in \mathcal{L}$ into agents $i \in \mathcal{I}$. Assume at time $t=0$ the map $I_0$ is an identity map so that the index $i$ coincides with $l$. (This assumes, just for simplicity, that  the numerability of agents is the same as that of locations.) Let $I_{t+1}=P \circ I_t$, $t\geq 0$.  Fix an agent $i$ and let $l$ be the unique solution to  $I_t(l)=i$. (As we constructed it, $I_{t+1}$ is a byjection.) Let $NBHD_t(i)=\{ i \in \mathcal{I} \mid i=I_t(l'), \; l-d \leq l'\leq l+d\}$. Then 

$$ prob(h^i_{t+1}=h' \mid h^i_t=h)= T_{h \: h'} ( [h_t^{i'}]_{i' \in NBHD(i)} ). $$  } 

\begin{equation} P_t \circ H_{t+1} =T(P_t \circ H_t)P_t \circ H_t. \label{locint} \end{equation}  
 
 The dynamical system is difficult to formally characterize, besides (possibly) an ergodicity result, with respect to initial conditions specifying, at  day $t=0$, a random allocation of  agents on evenly spaced locations in the City, all of them Susceptible, excepts for $A_0>0$ agents  who are exogenously infected Asymptomatics.  
 All our simulations, for all parameter values and initial conditions, converge to a unique ergodic distribution over the state space $h_t=\frac{1}{N}[S_t,A_t,Y_t,R_t,D_t] \in \Delta^{\mathcal{S}}$.
\\

\subsection{SIR and Spatial-SIR in Continuous Time and Space}

The SIR in continuous time model is the workhorse of the epidemiology literature, from \cite{kermack1927contribution}; see \cite{herben42hethcote} for a thorough mathematical presentation. It is representated by the following system of differential equations: 
\begin{eqnarray*} 
\frac{dI}{dt} &=& \beta S \frac{I}{N} -\rho I
\\
\frac{dR}{dt} &=&  \rho I
\\
N &=& S+I+R
\end{eqnarray*}

The SIR has been extended to a continuous space $s$ on some bounded domain $\Omega$; see e.g., \cite{chinviriyasit2010numerical} and \cite{wu2017epidemic}.  The Spatial-SIR  is then  represented by the following system of reaction-diffusion equations: 
\begin{eqnarray*} 
\frac{dI}{dt} -\alpha \Delta I &=& \beta S \frac{I}{N} -\rho I
\\
\frac{dR}{dt} \alpha \Delta I  &=&  \rho I
\\
N &=& \int_{\Omega} (S+I+R)ds, \; \; \mbox{for all } t
\end{eqnarray*}
where $\Delta$ is the Laplace operator, defined as the divergence $\nabla$ of the gradient $\nabla$, so that e.g., $\Delta I=\nabla^2 I$; and the boundary conditions, e.g., 
$$ \partial_{\eta} S=\partial_{\eta} I=\partial_{\eta} R=0 $$  where $\eta$ is the outward unit normal vector on the boundary of $\Omega$, $\partial \Omega$.



%\section{ Behavioral SIR \label{App:BehSIR} } 
%Consider a representative agent in SIR. Assume infected agents have no choice problem. Assume any susceptible agent can instead choose the number of contacts he/she has daily, $c$.  His/her contemporaneous  utility is increasing and concave in contacts, say $u(c)$. The present discounted utility he/she obtains if he/she gets infected is a constant $V_I$.   Discounting the future at a daily rate $\delta$, we can write the agent dynamic choice problem recursively as:

%\begin{equation} V(h)=\max_{c} u(c)+\delta \left[p\left(c, h\right)V_I +\left(1- p\left(c, h\right) \right) V(h')  \right];   \label{bellman} \end{equation} where $h=(I,R)$ is the state of the system,\footnote{We are abusing notation. We have defined $h=(S,I,R)$, but $S$ is residual.} $V(h)$ the value function, and $p(C(h),h)$ the probability a susceptible agents infected each one day, which in SIR is \begin{equation} p(c,h)=\pi c \frac{I}{N}. \label{prob} \end{equation} 
%The agent takes the state $h$ and $h'$ as given. 
%The first order conditions (necessary and sufficient under concavity) are: 

%\begin{equation} u'(c)= \delta \pi  \frac{I}{N} \left[V(h')-V_I \right] \label{focc} \end{equation}
%where $\left[V(h') -V_I\right]>0$ (to avoid trivialities) and $V(h)$ decreases with $I$ and increases with $R$ ($S$ is residual) [ this needs to be shown ]

%At the equilibrium dynamics of the system, $h'=(I', R')$ satisfies \begin{eqnarray} I' &=& I+\pi c (1-R-I)\frac{I}{N} \\ R'&=& R+\rho I \label{state} \end{eqnarray} 
%While the existence (and possibly uniqueness) of an equilibrium needs to be formulated as  fixed point  in the (infinite dimensional) space of sequences $(I,R)_t$, these sequences are observed in empirical work and hence this equilibrium condition need not be imposed in empirical analysis. 

%Consider now a perturbed problem in which $c=\alpha d \Psi$ and the agent chooses $\alpha$, at a convex cost $C(\alpha)$ such that $C(1)=0$. It is straighforward to show then that 
%\newtheorem{prop}{Proposition} 
%\begin{prop}\label{prop:pi_c} Given $(h, h')$ if $d$ is higher, $\alpha$ is lower, but $c= \alpha d \Psi$ is higher. \end{prop}
% \begin{proof} Convexity of $C(\alpha)$ implies that   $c=\alpha d \Psi$ must decrease. Convexity of , by \ref{focc} it must be that $V(h)$ is independent of $c$. But this contradicts  \ref{bellman}, as $\left[p\left(c, h\right)V_I +\left(1- p\left(c, h\right) \right) V(h')  \right]$ is decreasing in $\alpha$ and  $V_I<V(h)$). Then, back to \ref{focc}, by concavity of $u(c)$, $\alpha$ must be lower. Suppose now, by contradiction that $c=\alpha d \Psi $ is constant. Then, 
% $
% \left[p\left(c, h\right)V_I \right.$ $\left. +\left(1- p\left(c, h\right) \right) V(h') \right] 
% $ 
% decreases, as the probabilities are constant but the present value of $u$ is decreased as $c$ decreased. But then by \ref{focc} $c$ must be increasing. \end{proof}
% Notice that comparative statics with respect to some exogenous density parameter $d$ - whereby, e.g., the probability of contagion is $\pi d$ - is the same; that is $c$ decreases with $d$. Furthermore, it can be shown that $\pi d c$ increases. 

\section{Appendix: Additional Figures} \label{app:morefigures}

\subsection{Baseline model calibration}
\begin{figure}[H]
\caption{Growth rate of infections\label{fig-appgrowths}}
\begin{centering}
\begin{tabular}{c}
\includegraphics[width=3in]{figuresdir/nc5-app-baseline-lombardy-match.pdf}\tabularnewline
\end{tabular}
\par\end{centering}
\end{figure}

\subsection{How the model scales}
In Figure 
\ref{fig:2sizes}  we compare the  
progression of the contagion at days 10, 20, 30, 40, 50, and 70, between the baseline city 
and a city with four
times the population and the area (so that density is constant), and with four initial clusters of the same size as in the baseline
located in symmetric locations. Each panel reports on the right the geographical location of infections in the bigger city, on the bottom left the geographical location of infections in
the baseline (smaller) city, and on the top left the contagion rates. 

The progression of the infection is almost entirely
symmetric, barring minor effects due to the randomness of people's locations and movement. The top-right chart in each panel shows that both the fraction of active and total cases is nearly identical between the two Cities. 

\begin{figure}[H]
\caption{Rescaling a City}
\begin{threeparttable}
\centering
\makebox[\textwidth][c]{ %this is to center figure when larger than textwidth
    \begin{tabular}{cc}
         \includegraphics[width=0.5\textwidth]{figuresdir/sizecomp-day10.png}&\includegraphics[width=0.5\textwidth]{figuresdir/sizecomp-day20.png} \\
         \includegraphics[width=0.5\textwidth]{figuresdir/sizecomp-day30.png}&\includegraphics[width=0.5\textwidth]{figuresdir/sizecomp-day40.png} \\
         \includegraphics[width=0.5\textwidth]{figuresdir/sizecomp-day50.png}&\includegraphics[width=0.5\textwidth]{figuresdir/sizecomp-day70.png} \\
    \end{tabular}
    }
    \begin{tablenotes}
        \setlength{\itemindent}{-0.2em}
        \footnotesize
        \item
    The figure illustrates the progression of infections in two cities. From the top-left panel proceeding right and down: day 10, 20, 30, 40, 50, and 70. The small city (bottom right square in each panel) is our baseline city. The large city (large square) is a city of four times the area, four times the population, and 4 times the initial cluster of infections, placed in simmetric positions geographically. The top-left chart in each punel illustrates the fraction of active cases and total cases at day $t$. Yellow dots are the the active cases, green dots are the recovered cases (susceptibles are omitted). 
    \end{tablenotes}
\end{threeparttable}    
%\caption{Rescaling a city}
\label{fig:2sizes}
\end{figure}


\subsection{Lockdown policies}\label{app:lockdowns}
The role of local herd immunity appears very evident when comparing the effects of a lockdown policy (the typical Non-Pharmaceutical Intervention adopted in the SARS-CoC-2 epidemic) in SIR and in Spatial-SIR. 
%\subsection{Lockdowns}
Figure \ref{fig:policy-active} reports the dynamics of active cases under lockdowns restricting the movements of 30\% and 50\% of the population. The lockdowns are imposed when the fraction of active cases reach 10\% of the population and it is lifted when the fraction of active cases reaches 5\%. The left panel reports results from Spatial-SIR, the right panel from SIR. 
%The continuous lines in both panels report the results from the Spatial-SIR model without lockdown, the dotted lines from a lockdown policy that restricts movement of 30\% of the population, and the dashed lines a policy restricting movement of 50\% of the population.  
\begin{figure}[H]
    \caption{Comparison of models with 30\% and 50\% lockdown policies}
    \centering
    \label{fig:policy-active}
    \begin{threeparttable}
    \includegraphics[width=5in]{figuresdir/base_policy-10-5-active-.png}
    \begin{tablenotes}
        \footnotesize
        \setlength{\itemindent}{-0.2em}
        \item Left Panel: Spatial-SIR model, Right panel: SIR model. The lockdown is imposed when the fraction of active cases reaches 10\% of the population, and lifted when
        the fraction returns to 5\%
    \end{tablenotes}
    \end{threeparttable}
    
\end{figure}
Lockdowns have a smooth effect on the dynamics of active cases in SIR (right panel), reducing the peak from 70\% to 50\% (for the 50\% lockdown). Lifting the lockdown has minimal effects in SIR because, when active cases reach 5\%,  herd immunity is relatively far advanced. In Spatial-SIR, on the other hand,  the lockdown sets local herd immunity immediately in action (especially so the 50\% lockdown), dramatically reducing new cases (left panel). Cases however start surging as soon ad the lockdown is lifted, giving rise to the various waves/cycles  (especially so for the 50\% lockdown represented by the orange line). \newline

In Figures \ref{fig:policy-all-30} (resp. \ref{fig:policy-all-50}) 
we report, for both SIR and Spatial-SIR,  additional outcomes of the epidemic dynamics under these a policies. 

\begin{figure}[H]
    \caption{Comparison of models with 30\% lockdown}
    \makebox[\textwidth][c]{ %this is to center figure 
    \includegraphics[width=6.5in]{figuresdir/base_policy-10-5-30pc-comp.png}
    }
    \label{fig:policy-all-30}
\end{figure}

\begin{figure}[H]
    \caption{Comparison of models with 50\% lockdown}
        \makebox[\textwidth][c]{ %this is to center figure 
        \includegraphics[width=6.5in]{figuresdir/base_policy-10-5-50pc-comp.png}
    }
    \label{fig:policy-all-50}
\end{figure}

\subsection{Number of clusters}
In Figure \ref{fig:ncluster} we report the results of  simulations varying the number
of initial clusters from 0 to 20, in our otherwise baseline city, while keeping the number of initially infected agents constant. 
We observe that, with our calibrated parameters,  the effect of increasing the number of initial
clusters converges quite fast:
when there are five or more initial clusters, %the interaction of people's movement around the
%city with the location of the initial clusters of infections implies that 
increasing the 
number of initial clusters while keeping the number of initially infected the same, has no effect on the dynamics of the epidemics. 

\begin{figure}[H]
\caption{The effect of the number of clusters\label{fig:ncluster}}
\centering{}
\makebox[\textwidth][c]{ %this is to center figure when larger than textwidth
    \includegraphics[width=5.4in]{figuresdir/nc5-app-mclusters.pdf}
}
\end{figure}

\subsection{City size}
 In Figure \ref{fig:City-size} we report the results of 
    simulations varying the size of the City, in our otherwise baseline City,  while keeping the size of the initial outbreak of the infection and City density constant. 
We observe that, with our calibrated parameters,  the effect of increasing City size: the peak of active cases declines with size in a convex manner (less so the larger the city); the number of days it takes to reach the  peak  and the number of days to the stationary state (the end of the epidemic) both increases with size and do so with a slight concavity (less so the larger the city). 

\begin{figure}[H]
\caption{City size comparisons\label{fig:City-size}}
\centering{}
\makebox[\textwidth][c]{ %this is to center figure when larger than textwidth
    \includegraphics[width=5.5in]{figuresdir/nc5-app-sizes.pdf}
}
\end{figure}

\subsection{City density}
In Figure \ref{fig:density} we report the results of 
    simulations varying City density, in our otherwise baseline City.   
 We observe that, with our calibrated parameters: i) $(R^*+D^*)/N$ is increasing and concave in density; and the peak of $(A+Y)/N$ is also increasing and concave in density. In the right panel of  Figure \ref{fig:density} we see that the days it takes to reach the peak and the stationary state are decreasing and convex in density. % the effect of increasing City size: 
% Exploring more in detail the relationship between density and the various characteristics of the dynamics of the epidemic we focus on, in the right panel of  Figure \ref{fig:density} we see that indeed  %; see Figure \ref{fig:density}; left panel. 
% %\subsubsection{Social interactions}

% Density is a crucial determinant of the dynamics of the epidemic because, together with the contagion
% rates, it determines the average number of infections occurring on a given date. Increasing
% density while keeping the contagion radius the same increases the number of contacts that each infected individual has on a given day. Indeed, density is linearly related to the average number of contacts in the city; see Figure \ref{fig:density}, dotted line in the right panel). %Intuitively, the number of contacts, in our spatial model, has a similar role that $R_0$ has in the basic SIR, affecting both the transitional dynamics as well as the stationary state. 

% \begin{figure}[H]
% \caption{The effect of varying city density with constant population\label{fig:3density}}
% %codelocation sim-density.py
% \centering{}
% \makebox[\textwidth][c]{ %this is to center figure when larger than textwidth
%     \includegraphics[width=5.5in]{figuresdir/nc5-3densities.pdf} 
% }
% \end{figure}


\begin{figure}[H]
\caption{The effect of varying city density with constant population\label{fig:density}}
%codelocation sim-density.py
\centering{}
\makebox[\textwidth][c]{ %this is to center figure when larger than textwidth
    \includegraphics[width=5.5in]{figuresdir/nc5-app-densities.pdf} 
}
\end{figure}

\newpage
\bibliographystyle{aer}
\bibliography{covid}



\end{document}